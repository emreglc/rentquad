\chapter*{Özet}
\addcontentsline{toc}{chapter}{Özet}

%% Edit below this line
Bu tez, kullanıcıların otonom bir quadricycle filosunu temassız biçimde kiralamasını sağlayan mobil uygulama, bulut tabanlı arka uç ve araç elektroniklerinden oluşan tümleşik bir sistemi sunmaktadır. Kent içi mikro mobilite alanında yaşanan erişilebilirlik, güvenlik ve operasyon maliyeti sorunları; rezervasyon, QR kodla sürüş başlatma, anlık harita takibi ve otomatik ödeme kapanışı özelliklerini aynı iş akışında birleştiren akıllı çözümlere ihtiyaç duymaktadır. Geliştirilen Expo React Native uygulaması, Supabase servisleri ve quadricycle üzerindeki IoT modülleri üzerinden haberleşerek aracın kilitlenmesi, telemetri takibi ve kullanıcı etkileşimi gibi kritik süreçleri uçtan uca yönetmektedir.

Çalışmada gereksinim analizi, katmanlı sistem mimarisi, harita destekli kullanım senaryoları ve kiralama akışını yöneten durum makinesi ayrıntılı biçimde açıklanmıştır. Uygulama geliştirmesi, tümleşik mimarinin teknik olarak gerçeklenebilir ve ölçeklenebilir olduğunu ortaya koymuştur. Sonuçlar, önerilen platformun Gebze Teknik Üniversitesi ölçütlerine uygun güvenli bir otonom quadricycle paylaşım sistemi çerçevesi oluşturduğunu göstermektedir.

%% Until here
\vfill
%% Edit after {Anahtar Kelimeler:}
\textbf{Anahtar Kelimeler:} otonom araç kiralama, quadricycle, React Native, Supabase, IoT telemetrisi.
\clearpage